\documentclass[11pt]{article} 
\usepackage{graphicx}
\usepackage{hyperref}
\usepackage{amsmath}
\usepackage{verbatim}
\usepackage{fancyvrb}
\usepackage{natbib}
\usepackage{color}
\hypersetup{
  % bookmarks=true,         % show bookmarks bar?
    unicode=false,          % non-Latin characters in Acrobat’s bookmarks
    pdftoolbar=true,        % show Acrobat’s toolbar?
    pdfmenubar=true,        % show Acrobat’s menu?
    pdffitwindow=false,     % window fit to page when opened
    pdfstartview={FitH},    % fits the width of the page to the window
    pdftitle={Pulmonary Vessel Segmentation Notes},
    pdfauthor={Wei Liu},     % author
    pdfsubject={Reading notes},   % subject of the document
    pdfcreator={Wei Liu},   % creator of the document
    pdfproducer={Producer}, % producer of the document
    pdfkeywords={Pulmonary vessel, segmentation}, % list of keywords
    pdfnewwindow=true,      % links in new window
    colorlinks= true,       % false: boxed links; true: colored links
    linkcolor=blue,          % color of internal links
    citecolor=blue,        % color of links to bibliography
    filecolor=magenta,      % color of file links
    urlcolor=cyan           % color of external links
}

\setlength{\oddsidemargin}{0 in}
\setlength{\evensidemargin}{0 in}
\setlength{\topmargin}{-0.6 in}
\setlength{\textwidth}{6.5 in}
\setlength{\textheight}{9 in}
\setlength{\headsep}{0.5 in}
\setlength{\parindent}{0 in}
\setlength{\parskip}{0.1 in}

\begin{document}
\title{Pulmonary Vessel Segmentation Notes}
\author{Wei Liu}
\maketitle

\section{Lung Extraction Pipeline}
\begin{itemize}
\item Threshold the CT image at -2000 to get a \emph{cylinder} mask. The -2000
  value may change across patients?

\begin{Verbatim}[frame=single]
fslmaths /scratch/datasets/PE/PE000919.nii.gz -add 2000 -thr 0 -bin
PE919/round_mask.nii.gz
\end{Verbatim}

\item First run GMM with 3 components: lung, tissue, and bones. Initial
mean set to -800, 0, 1000, and initial standard deviation of GMM set to 100
for all components.  In some cases, we need to run GMM with 2 components.
\begin{Verbatim}[frame=single]
gmm --input RV01.nii.gz --seg seg.nii.gz --mask round_mask.nii.gz --ncomp 3
--mean -800 0 100 --sigma 100 100 100 --prop 0.33 0.33 0.33 --maxit 30
\end{Verbatim}
\item Convert the GMM label map into a component map, so the non-connected
  regions with same GMM labels are assigned different component labels.
\begin{Verbatim}[frame=single]
sitkfuncs.extract_comp('RV01/gmm_seg.nii.gz', 'RV01/cc.nii.gz')
\end{Verbatim}
\item Use ITK-SNAP to check the label corresponding to the left and right
  lung. If left and right lungs are connected, identify only one
  label. Otherwise, identify both labels. 

\item Extract the lung component and output a binary volume.

\begin{Verbatim}[frame=single]
sitkfuncs.extract_lung('RV01/cc.nii.gz', [1], 'RV01/lung.nii.gz', 10)
\end{Verbatim}

\item Optionally, fill hole in the lung mask with the fillhole filter. ITK
  fillhole filter has a bug, that it only fill holes in a axial slides. That
  is, the filter applies only on 2D slide, instead of on 3D volume. The below
  command call ITK filter, so it does the same thing. 

\begin{Verbatim}[frame=single]
fillhole_filter -i lung.nii.gz -o lung.nii.gz
\end{Verbatim}

\end{itemize}

\section{Vessel Extraction}
\begin{itemize}
\item Manually define seed regions.

\item Estimate the density and get a density map, which will be used as the
  speed map of fast marching method. 
\begin{Verbatim}[frame=single]
est_density -i RV01.nii.gz -e seeds.nii.gz -m lung.nii.gz 
-o density.nii.gz
\end{Verbatim}
The standard deviation parameter of the density estimation routine controls
how much belief the user should give to the seed region. With a small
deviation, only the voxels with intensity close to the mean intensity of the
seed regions will have larger density value. With a larger standard deviation,
the voxel have non-zero density value even its intensity is quite different
from the mean intensity. Therefore, the standard deviation parameter can be
seen as a regularization. A larger value will make the histogram of the
density map more flat, and make Vessel and non-vessel voxels density more
similar. accordingly, fast marching have more chance to propagate to
non-vessel regions.

  \item Run the fast marching method:
\begin{Verbatim}[frame=single]
fmm_upwind -p density.nii.gz -e seeds.nii.gz -m lung.nii.gz -t 500 
-o fmm_out.nii.gz
\end{Verbatim}

\item Inverse the value of the \emph{time of visit} map to get a heat map,
  where larger values represent vessels. 
\begin{Verbatim}[frame=single]
inverse_distmap -i fmm_out.nii.gz -m lung.nii.gz 
-o vessel.nii.gz -x 500
\end{Verbatim}
\end{itemize}

\bibliographystyle{plainnat}
\bibliography{/home/weiliu/projects/myref}
\end{document}
